\chapter{Introduction}
\label{Introduction}

\section{What is \textsc{Line}?}
\label{what-is-line?}
\textsc{Line} is an engine for system performance and reliability evaluation based on queueing theory and stochastic modeling. Systems analyzed with \textsc{Line} may either be software applications, business processes, computer networks, or else. \textsc{Line} decomposes a high-level system model into one or more stochastic models, typically extended queueing networks, that are subsequently analyzed for performance and reliability metrics using either numerical algorithms or simulation.

A key feature of \textsc{Line} is that the engine decouples the model description from the solvers used for its solution. That is, the engine implements model-to-model transformations that automatically translate the model specification into the input format (or data structure) accepted by the target solver. External solvers supported by \textsc{Line} include Java Modelling Tools (JMT; \url{http://jmt.sf.net}) and LQNS (\url{http://www.sce.carleton.ca/rads/lqns/}). Native model solvers are based on formalisms and techniques such as:
\begin{itemize}
\item Continuous-time Markov chains (\texttt{CTMC})
\item Fluid ordinary differential equations (\texttt{FLUID})
\item Matrix analytic methods (\texttt{MAM})
\item Normalizing constant analysis (\texttt{NC})
\item Mean-value analysis (\texttt{MVA})
\item Stochastic simulation (\texttt{SSA})
\end{itemize}
Each solver encodes a general solution paradigm and can implement both exact and approximate analysis methods. For example, the \texttt{MVA} solver implements both exact mean value analysis (MVA) and approximate mean value analysis (AMVA). The offered methods typically differ for accuracy, computational cost, and the subset of model features they support. A special solver (\texttt{AUTO}) is supplied that provides an automated recommendation on which solver to use for a given model.

The above techniques can be applied to models specified in the following formats:
\begin{itemize}
\item {\em \textsc{Line} modeling language (MATLAB script format)}. This is a MATLAB-based object-oriented language designed to resemble the abstractions available in JMT's queueing network simulator (JSIM). Among the main benefits of this language is that \textsc{Line} models can be exported to, and visualized with, JSIMgraph.
\item {\em Layered queueing network models (LQNS XML format)}. \textsc{Line} is able to solve a sub-class of layered queueing network models, provided that they are specified using the XML metamodel of the LQNS solver. %\footnote{\url{https://github.com/layeredqueuing/}}. %\textsc{Line} has been successfully used to parse LQN models generated by Palladio Bench's default PCM2LQN transformation and by the Tulsa UML2LQN transformation\footnote{\url{https://github.com/dice-project/DICE-Tulsa}}~\cite{LiAZCP17}.
%\item {\em Business Process Modeling Notation (BPMN)} \textsc{Line} is able to import and solve basic BPMN 2.0 collaboration diagrams, extended with performance annotations.
\item {\em JMT simulation models (JSIMg, JSIMw formats)}. \textsc{Line} is able to import and solve queueing network models specified using JMT's simulation tools, namely JSIMgraph and JSIMwiz.
\item {\em Performance Model Interchange Format (PMIF XML format)}. \textsc{Line} is able to import and solve closed queueing network models specified using PMIF v1.0.
\end{itemize}

\section{Obtaining the latest release}
\label{obtaining-the-latest-release}
This document contains the user manual for \textsc{Line} version 2.0.0-ALPHA, which can be obtained from:
\begin{center}
{\url{https://github.com/line-solver/line/}}
\end{center}
\noindent \textsc{Line} 2.0.0-ALPHA has been tested on MATLAB R2017b and later releases and requires the \emph{Statistics and Machine Learning Toolbox}. If you are interested to obtain \textsc{Line} as a JAR, or as an executable distribution for any of the operating systems supported by the MATLAB Compiler Runtime (MCR), please contact the maintainer.

\section{Installation and demos}
\label{installation-and-demos}
This is the fastest way to get started with \textsc{Line}:
\begin{enumerate}
\item Download/clone the latest release:
\begin{itemize}
%\item Stable release (zip/tar.gz): \url{https://github.com/line-solver/line/releases}
\item Git repository: \url{https://github.com/line-solver/line/}
\end{itemize}
Ensure that files are available in the chosen installation folder.

\item Start MATLAB and change the active directory to the installation folder. Then add all sub-folders to the MATLAB path
\begin{lstlisting}
addpath(genpath(pwd))
\end{lstlisting}
\item Run the demonstrators using
\begin{lstlisting}
allExamples
\end{lstlisting}
\end{enumerate}

\section{Getting help}
For bugs or feature requests, please use: \url{https://github.com/line-solver/line/issues}

\section{References}
\label{references}
\noindent To cite \textsc{Line}, we recommend to reference:
\begin{itemize}
\item \noindent J. F. P\'erez and G. Casale. ``LINE: Evaluating Software Applications in Unreliable Environments'', in {\em IEEE Transactions on Reliability}, Volume 66, Issue 3, pages 837-853, Feb 2017. {\em This paper introduces \textsc{Line} version 1.0.0}.
\end{itemize}

\noindent The following papers discuss recent applications of \textsc{Line}:
\begin{itemize}
\item \noindent C. Li and G. Casale. ``Performance-Aware Refactoring of Cloud-based Big Data Applications'', in { Proceedings of 10th IEEE/ACM International Conference on Utility and Cloud Computing}, 2017. {\em This paper uses \textsc{Line} to model stream processing systems}.

\item \noindent D. J. Dubois, G. Casale. ``OptiSpot: minimizing application deployment cost using spot cloud resources'', in {\em Cluster Computing}, Volume 19, Issue 2, pages 893-909, 2016. {\em This paper uses \textsc{Line} to determine bidding costs in spot VMs}.

\item \noindent R. Osman, J. F. P\'erez, and G. Casale. ``Quantifying the Impact of Replication on the Quality-of-Service in Cloud Databases'. {Proceedings of the IEEE International Conference on Software Quality, Reliability and Security (QRS)}, 286-297, 2016. {\em This paper uses \textsc{Line} to model the Amazon RDS database}.

\item C. M{\"{u}}ller, P. Rygielski, S. Spinner, and S. Kounev. {Enabling Fluid Analysis for Queueing Petri Nets via Model Transformation}, {Electr. Notes Theor. Comput. Sci}, {327}, {71--91}, {2016}. {\em This paper uses \textsc{Line} to analyze Descartes models used in software engineering}.

\item  \noindent J. F. P\'erez and G. Casale. ``Assessing SLA compliance from Palladio component models,'' in {Proceedings of the 2nd Workshop on Management of resources and services in Cloud and Sky computing (MICAS)}, IEEE Press, 2013. {\em This paper uses \textsc{Line} to analyze Palladio component models used in model-driven software engineering}.
\end{itemize}

\section{Contact}
\noindent Project coordinator and maintainer contact:
\begin{verbatim}
Giuliano Casale
Department of Computing
Imperial College London
180 Queen's Gate
SW7 2AZ, London, UK.
\end{verbatim}
\noindent \texttt{Web:} \url{http://wp.doc.ic.ac.uk/gcasale/}

\section{Copyright and license}
Copyright Imperial College London (2015-Present). \textsc{Line} 2.0.0-ALPHA is freeware, but closed-source, and released under the 3-clause BSD license. Additional licensing information is available in the license file: \url{https://raw.githubusercontent.com/line-solver/line/master/LICENSE}. License files of third-party libraries are placed under the \texttt{lib/} directory.

\section{Acknowledgement}
\textsc{Line} has been partially funded by the European Commission grants FP7-318484 (MODAClouds), H2020-644869 (DICE), and by the EPSRC grant EP/M009211/1 (OptiMAM).


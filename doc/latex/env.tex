\chapter{Random environments}
\label{Random-environments}
Systems modeled with \textsc{Line} can be described as operating in an environment whose state affects the way the system operates. To distinguish the states of the environment from the ones of the system within it, we shall refer to the former as the environment {\em stages}.  In particular, \textsc{Line} 2.0.0-ALPHA supports the definition of a class of random environments subject to three assumptions:
\begin{itemize}
\item The stage of the environment evolves independently of the state of the system.
\item The dynamics of the environment stage can be described by a continuous-time Markov chain.
\item The topology of the system is independent of the environment stage.
\end{itemize}
The above definitions are in particular appropriate to describe systems whose input parameters change with the environment stage. For example, an environment with two stages, say normal load and peak load, may differ for the number of servers that are available in a queueing station, i.e., the system controller may add more servers during peak load. Upon a stage change in the environment, the model parameters will instantaneously change, but the state reached during the previous stage will be used to initialize the system in the new stage.

Although in a number of cases the system performance may be similar to a weighted combination of the average performance in each stage, this is not true in general, especially if the system dynamic (i.e., the rate at which jobs arrived and get served) and the environment dynamic (i.e., the rate at which the environment changes active stage) happen at similar timescales~\cite{CasTH14}.

\section{Environment object definition}
\label{environment-object-definition}
\subsection{Specifying the environment transitions}
To specify an environment, it is sufficient to define a cell array with entries describing the distribution of time before the environment jumps to a given target state. For example
\begin{lstlisting}
env{1,1} = Exp(0);
env{1,2} = Exp(1);
env{2,1} = Exp(1);
env{2,2} = Exp(0);
\end{lstlisting}
describes an environment consisting of two stage, where the time before a transition to the other stage is exponential with unit rate. If we were to set instead
\begin{lstlisting}
env{2,2} = Erlang.fitMeanAndOrder(1,2);
\end{lstlisting}
this would cause a race condition between two distributions in stage two: the exponential transition back to stage 1, and the Erlang-2 distributed transition with unit rate that remains in stage 2. The latter means that periodically the system will be re-initialized in stage 2, meaning that jobs in execution at a server are required all to restart execution.

In \textsc{Line}, an environment is internally described by a Markov renewal process (MRP) with transition times belonging to the \texttt{PhaseType} class. A MRP is similar to a Markov chain, but state transitions are not restricted to be exponential. Although the time spent in each state of the MRP is not exponential, the MRP can be easily transformed into an equivalent continuous-time Markov chain (CTMC) to enable analysis, a task that \textsc{Line} performs automatically. In the example above, the underpinning CTMC will therefore consider the distribution of the minimum between the exponential and the Erlang-2 distribution, in order to decide the next stage transition.

State space explosion may occur in the definition of an environment if the user specifies a large number of non-exponential transition. For example, a race condition among $n$ Erlang-2 distribution translates at the level of the CTMC into a state space with $2^n$ states. In such situations, it is recommended to replace some of the distributions with exponential ones.

\subsection{Specifying the system models}
\textsc{Line} places loose assumptions in the way the system should be described in each stage. It is just expected that the user supplies a model object, either a \texttt{Network} or a \texttt{LayeredNetwork}, in each stage, and that a transient analysis method is available in the chosen solver, a requirement fulfilled for example by \texttt{SolverFluid}.

However, we note that the model definition can be somewhat simplified if the user describes the system model in a separate MATLAB function, accepting the stage-specific parameters in input to the function. This enables reuse of the system topology across stages, while creating independent model objects. An example of this specification style is given in \texttt{example\_randomEnvironment\_1.m} under \textsc{Line}'s example folder.

\section{Solvers}
The steady-state analysis of a system in a random environment is carried out in \textsc{Line} using the blending method~\cite{CasTH14}, which is an iterative algorithm leveraging the transient solution of the model. In essence, the model looks at the \emph{average} state of the system at the instant of each stage transition, and upon restarting the system in the new stage re-initializes it from this average value. This algorithm is implemented  in \textsc{Line} by the \texttt{SolverEnv} class, which is described next.

\subsection{\texttt{ENV}}
 The \texttt{SolverEnv} class applies the blending algorithm by iteratively carrying out a transient analysis of each system model in each environment stage, and probabilistically weighting the solution to extract the steady-state behavior of the system.

 As in the transient analysis of \texttt{Network} objects, \textsc{Line} does not supply a method to obtain mean response times, since Little's law does not hold in the transient regime. To obtain the mean queue-length, utilization and throughput of the system one can call as usual the \texttt{getAvg} method on the \texttt{SolverEnv} object, e.g.,
\begin{lstlisting}
models = {model1, model2, model3, model4};
envSolver = SolverEnv(models, env, @SolverFluid,options);
[QN,UN,TN] = envSolver.getAvg()
\end{lstlisting}
Note that as model complexity grows, the number of iterations required by the blending algorithm to converge may grow large. In such cases, the \texttt{options.iter\_max} option may be used to bound the maximum analysis time. 